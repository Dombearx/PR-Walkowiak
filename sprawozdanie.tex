 \documentclass{article}

\usepackage{polski}
\usepackage[utf8]{inputenc}
\usepackage{graphicx}
\usepackage{float}
\usepackage{listings} 
\renewcommand{\lstlistingname}{Funkcja}

\lstset{language=C++,
numbers=left}


\author{Dominik Doberski, Artur Olejnik}
\title{\huge Sprawozdanie \\ Przetwarzanie równoległe\\ \Huge Projekt 1 OMP
}



\begin{document}
\maketitle

\section{Wstęp}
\subsection{Temat}
Mnożenie macierzy - porównanie efektywności metod --
\begin{itemize}
\item 3 pętle - kolejność pętli: jik, podział pracy przed pętlą 1
\item 6 pętli -kolejność pętli: zewnętrznych ijk, wewnętrznych: ii,kk,jj podział pracy przed pętlą 2.
\end{itemize}
\subsection{Autorzy}
\begin{minipage}[t]{0.3\textwidth}
% Pierwsza kolumna (średnia)
Artur Olejnik\\
Dominik Doberski
\end{minipage}
\begin{minipage}[t]{0.15\textwidth}
% Druga kolumna (mniejsza)
122402\\
132207
\end{minipage}
\begin{minipage}[t]{0.55\textwidth}
% Trzecia kolumna (mniejsza)
artur.olejnik@student.put.poznan.pl\\
dominik.doberski@student.put.poznan.pl
\end{minipage}
\\\\\\
Grupa dziekańska I3\\
Termin zajęć: poniedziałek 16:50 
\subsection{Opis systemu obliczeniowego}
Do wykonania pomiarów wykorzystano program CodeXL. Testy zrealizowano na jednostce obliczeniowej o poniższej specyfikacji:
\begin{itemize}
\item Nazwa: AMD FX-6300 Black Edition
\item Częstotliwość taktowania: 3.5 GHz (max 3.8 GHz)
\item Liczba procesorów fizycznych: 6
\item Liczba procesorów logicznych: 6
\item Liczba uruchamianych w systemie wątków: 6
\item Wielkość i organizacja pamięci podręcznych procesora:\\
L1 dane: 6x16KB\\
L1 instrukcje: 3x64KB\\
L2: 3x2MB\\
L3: 8MB\\
\end{itemize}

\section{Analiza algorytmu}
\subsection{Analiza kodu}
\begin{lstlisting}[
caption={3 pętle -- jik},
label=first,
firstnumber=1]
void multiply_matrices_JIK() {
#pragma omp parallel
#pragma omp for 
 for (int j = 0; j < ROWS; j++){
  for (int i = 0; i < ROWS; i++){
   for (int k = 0; k < ROWS; k++){
    matrix_r[i][j] += matrix_a[i][k] * matrix_b[k][j];
   }
  }
 }
}
\end{lstlisting}
W powyższej funkcji pętla w 6. wierszu iteruje po kolumnach macierzy A i po wierszach macierzy B. Iloczyn tych dwóch elementów jest dodawany do pojedynczego elementu macierzy R znajdującego się w i-tym wierszu i j-tej kolumnie, które są wyznaczane w pętlach odpowiednio 5. i 4. wierszu. Złożoność algorytmu wynosi $O(n^3)$. Dyrektywy OpenMP w 2. i 3. wierszu wykorzystano do uruchomienia kodu na wielu wątkach, oraz podzielenia przetwarzania pętli na te wątki.
\begin{lstlisting}[
caption={6 pętli -- zewnętrzne ijk, wewnętrzne ii,kk,jj},
label=first,
firstnumber=12]
void multiply_matrices_6(int R){
#pragma omp parallel
 for (int i = 0; i < ROWS; i += R) {
#pragma omp for
  for (int j = 0; j < ROWS; j += R) {
   for (int k = 0; k < ROWS; k += R) {
    for (int ii = i; ii < i + R; ii++) {
     for (int kk = k; kk < k + R; kk++) {
      for (int jj = j; jj < j + R; jj++) {
       matrix_r[ii][jj] += matrix_a[ii][kk] * matrix_b[kk][jj];
      }
     }
    }
   }
  }
 }	
}
\end{lstlisting}



\section{Analiza przetwarzania}
\subsection{Przebieg}
asdf
\subsection{Wyniki}
asdf
\subsection{Prezentacja wzorców}
asdf
\subsection{Miary przetwarzania}
asdf

\section{Wnioski}
Chujowy projekt.

\subsection{Wypunktowania}
To jest wypunktowanie:
\begin{itemize}
\item punkt
\item punkt
\item punkt
\end{itemize}
\subsection{Wyliczenie}
To jest wyliczenie:
\begin{enumerate}
\item pierwszy,
\item drugi,
\item trzeci.
\end{enumerate}
\subsection{Wykres}
Na Rysunku~\ref{fig:wykres} znajduje się przykładowy wykres.
\subsection{Wzory}
\begin{equation}
2 + 2 = 4
\end{equation}
\begin{equation}
E = mc^2
\end{equation}
\begin{equation}
\left[- \frac{\hbar^2}{2M} + V(X) \right] \Psi(x)=  \mathcal{E} \Psi(x)
\end{equation}

\section{Podsumowanie}

\end{document}
